%===============================================================================
% $Id: ifacconf.tex 19 2011-10-27 09:32:13Z jpuente $  
% Template for IFAC meeting papers
% Copyright (c) 2007-2008 International Federation of Automatic Control
%===============================================================================
\documentclass[a4paper]{ifacconf}
\usepackage{graphicx,amsmath,url}      % include this line if your document contains figures
\usepackage[round]{natbib}             % required for bibliography
\usepackage{siunitx}
%===============================================================================


% ===============================================================
% Choose the language of the manuscript.
% If in English, choose 
% \def\portugues{0} 
%
% If in Portuguese or Spanish, choose
% \def\portugues{1} 
%
% Note that, if you are writing in Spanish, you need additional 
% adjusts in some parts of the text, which have been put in Portuguese only.
\def\portugues{1} 
% ===============================================================

% If the above line is commented, it is assumed manuscript in English:
\ifx\portugues\undefined
\def\portugues{0}
\fi


\if\portugues0
   \usepackage[english]{babel}
  \else
   \usepackage[spanish,brazil,english]{babel}
\fi

  

\usepackage[T1]{fontenc}
%\usepackage{inputenc}

\usepackage[utf8]{inputenc}

\usepackage{ae}


\if\portugues1
% =====================================================================
% =====================================================================
% If the manuscript is in Spanish, please change the texts adequatelly.
% You may also add other definitions in this part.
 \newtheorem{teorema}[thm]{{\em Teorema}}{ }
 \newtheorem{lema}[thm]{{\em Lema}}{ }
 \newtheorem{corolario}[thm]{{\em Corolário}}{ }
 \newenvironment{prova}{{\bf Prova.}}{ }
% ===============================================================
\fi

\begin{document}
	
	
\if\portugues1

% =====================================================================
% =====================================================================
% USE THIS PART IF THE TEXT IS IN PORTUGUES OR SPANISH
% =====================================================================
% If the manuscript is in Spanish, please change the texts adequately.
% =====================================================================
% 
\selectlanguage{brazil}
	
\begin{frontmatter}

\title{Desenvolvimento, Implementação e Estudo de Caso de um Software de Tempo Real com Aprendizado de Máquina
para Detecção Automática de Falhas em Motores Elétricos de Indução} 
% Title, preferably not more than 10 words.

% \thanks[footnoteinfo]{Reconhecimento do suporte financeiro deve vir nesta nota de rodapé.}


\author[First]{Piaia, G. A.} 
\author[Second]{Hoffmann, G.} 
\author[Third]{Weber, L. O.}
\author[Fourth]{Figueiredo, R. M.}

\address[First]{Programa de Pós-Graduação em Engenharia Elétrica, Universidade do Vale do Rio Dos Sinos - 
UNISINOS, RS \& Floki Desenvolvimento de Sistemas Industriais, RS (e-mail: piaia@flokisys.com).}
\address[Second]{Floki Desenvolvimento de Sistemas Industriais, RS (e-mail: giu@flokisys.com).}
\address[Third]{Floki Desenvolvimento de Sistemas Industriais, RS (e-mail: weber@flokisys.com).}
\address[Fourth]{Programa de Pós-Graduação em Engenharia Elétrica, Universidade do Vale do Rio Dos Sinos - 
UNISINOS, RS (e-mail: marquesf@unisinos.br).}



\selectlanguage{english}
\renewcommand{\abstractname}{{\bf Abstract:~}}
\begin{abstract}                % Abstract of not more than 250 words.
                        This work presents a solution that integrates software and hardware for solve the aforementioned problem,
                        including in real time and that learns the behaviour equipment, suggesting regions of alert and
                        danger, serving as a tool for taking decisions. For the development, the physical quantities of
                        vibration were used, together with with signal processing techniques and machine learning. 
                        The solution was able to detected in advance in one case, and not the other, demonstrate that the equipment was in
                        severe condition.

\vskip 1mm% não altere esse espaçamento
\selectlanguage{brazil}
{\noindent \bf Resumo}: O presente trabalho propõe uma solução que integra software e hardware para
                        resolver o problema antes citado, inclusive em tempo real e que aprende o comportamento
                        do equipamento, sugerindo regiões de alerta e perigo, servindo de ferramenta para tomada de
                        decisões. Para o desenvolvimento, as grandezas físicas de vibração foram utilizadas, juntamente
                        com técnicas de processamento de sinais e machine learning (aprendizado de máquina). A solução 
                        foi empregada um estudo de caso em empresas de diferentes ramos da indústria, onde
                        o sistema detectou com antecedência em um dos casos, e no outro, demonstrou que o equipamento se
                        encontrava em estado severo.
\end{abstract}

\selectlanguage{english}


\begin{keyword}
  Vibration; Fault Analysis; Vibration and Temperature Analysis; Fault Detection; Real Time Monitoring; Machine Learning;
  Induction Motor.

\vskip 1mm% não altere esse espaçamento
\selectlanguage{brazil}
{\noindent\it Palavras-chaves:} Análise de Vibração e temperatura; Detecção de Falhas; Monitoramento em tempo real; Motores elétricos de indução; 
Aprendizado de Máquina.
\end{keyword}


\selectlanguage{brazil}


\end{frontmatter}
\else
% ===============================================================
% ===============================================================
% USE THIS PART IF THE TEXT IS IN ENGLISH
% ===============================================================
% ===============================================================
% 

\begin{frontmatter}

\title{Style for SBA Conferences \& Symposia: Use Title Case for
  Paper Title\thanksref{footnoteinfo}} 
% Title, preferably not more than 10 words.

\thanks[footnoteinfo]{Sponsor and financial support acknowledgment
goes here. Paper titles should be written in uppercase and lowercase
letters, not all uppercase.}

\author[First]{First A. Author} 
\author[Second]{Second B. Author, Jr.} 
\author[Third]{Third C. Author}


\address[First]{Faculdade de Engenharia Elétrica, Universidade do Triângulo, MG, (e-mail: autor1@faceg@univt.br).}
\address[Second]{Faculdade de Engenharia de Controle \& Automação, Universidade do Futuro, RJ (e-mail: autor2@feca.unifutu.rj)}
\address[Third]{Electrical Engineering Department, 
   Seoul National University, Seoul, Korea, (e-mail: author3@snu.ac.kr)}
   
\renewcommand{\abstractname}{{\bf Abstract:~}}   
   
\begin{abstract}                % Abstract of not more than 250 words.
These instructions give you guidelines for preparing papers for IFAC
technical meetings. Please use this document as a template to prepare
your manuscript. For submission guidelines, follow instructions on
paper submission system as well as the event website.
\end{abstract}

\begin{keyword}
Five to ten keywords, preferably chosen from the IFAC keyword list.
\end{keyword}

\end{frontmatter}
\fi

%===============================================================================
%===============================================================================
%===============================================================================


\section{Introdução}

\begin{figure}[h!]
  \begin{center}
      \includegraphics[scale=0.53, page=1]{../metodologia/img/fluxo_layout.pdf}
  \end{center}
  \caption{Diagrama de integração  entre sensores e \textit{software}. Fonte: Adaptado pelo autor.}
  \label{fig:fluxo_integracao}
\end{figure}


\begin{figure}[h!]
  \begin{center}
      \includegraphics[scale=.35]{../referencial/img/misadraw_analog_p2.png}
  \end{center}
  \caption{Ilustração com os tipos de desalinhamento. Fonte: \cite{Sopcik2019}.}
  \label{fig:misadraw_analog_p2}
\end{figure}

%===============================================================================

\section{Solução Proposta}


\begin{figure}[h!]
  \begin{center}
      \includegraphics[scale=0.53, page=2]{../metodologia/img/fluxo_layout.pdf}
  \end{center}
  \caption{Fluxo de acesso do \textit{software}. Fonte: Elaborado pelo autor.} 
  \label{fig:fluxo_software}
\end{figure}


%===============================================================================

\section{Metodologia}



\begin{figure}[h!]
  \begin{center}
      \includegraphics[scale=0.8, page=3]{../metodologia/img/software.pdf}
      \caption{Metodologia do Projeto. Fonte: Elaborado pelo autor.}
  \end{center} 
  \label{fig:metodologia}
\end{figure}

\begin{figure}[h!]
  \caption{Tarefas de Aquisição e ML. Fonte: Elaborado pelo autor.}
  \begin{center}
      \includegraphics[scale=0.8, page=1]{../metodologia/img/software.pdf}
  \end{center}
  \label{fig:tarefa_aq_ml}
\end{figure}


\begin{table}[h!]
  \caption{Especificações do Sensor QM30VT2. Fonte: Elaborado pelo Autor.}
  \label{tab:espec_sensor}
  \centering
  \begin{minipage}{.33\textwidth}
    \begin{tabular*}{\textwidth}{c|c}
      \hline
      Especificação            & Valores                                     \\ \hline
      \hline
      Intervalo de medição     &  0 a $\SI{46}{\milli\metre\per\second}$     \\
      Largura de banda         &  10 a $\SI{4}{\kilo\hertz}$                 \\ 
      Acurácia                 &  $\mp 10 \%$ a $\SI{25}{\celsius}$          \\
      Amostragem               &  $\SI{20}{\kilo\hertz}$                     \\
      Tempo de Amostragem      &  $\SI{0.4}{\second}$                        \\
      Grau de Certificação     &  IP67                                       \\ \hline
    \end{tabular*}
  \end{minipage}
\end{table}


\subsection{Estdo de Caso}


\begin{figure}[h!]
  \begin{center}
      \includegraphics[scale=1]{../metodologia/img/exaustor.png}
  \end{center}
  \caption{Fotografia externa de um exaustor industrial. Fonte: Fornecida pela empresa.}
  \label{fig:exautor}
\end{figure}

\begin{figure}[h!]
  \begin{center}
      \includegraphics[scale=1]{../metodologia/img/sensor_exaustor.jpg}
  \end{center}
  \caption{Fotografia externa do sensor instalado em um exaustor industrial. Fonte: Forncecida pela empresa.}
  \label{fig:sensor_exaustor}
\end{figure}

%===============================================================================

\section{Resultados}

\begin{table}[h!]
  \begin{center}
    \caption{Margens.}\label{tb:margins}
    \begin{tabular}{cccc}
      Página & Topo & Baixo & Esquerda/Direita \\\hline
      Primeira & 3,5 & 2,5 & 1,5 \\
      Demais & 2,5 & 2,5 & 1,5 \\ \hline
    \end{tabular}
  \end{center}
\end{table}


\begin{figure}[h!]
  \begin{center}
      \includegraphics[scale=0.55, page=1]{../resultados/img/resultados.pdf}
  \end{center}
  \caption{Captura de tela do sistema instalado no exaustor. Fonte: Elaborado pelo Autor.}
  \label{fig:exaustor_1}
\end{figure}

\begin{figure}[h!]
  \begin{center}
      \includegraphics[scale=0.55, page=2]{../resultados/img/resultados.pdf}
  \end{center}
  \caption{Velocidade RMS [\SI{}{\milli\metre\per\second}] coletados do eixo x (A) e z (B). Fonte: Elaborado pelo Autor.}
  \label{fig:exaustor_xz}
\end{figure}

\begin{figure}[h!]
  \begin{center}
      \includegraphics[scale=0.55, page=3]{../resultados/img/resultados.pdf}
  \end{center}
  \caption{Temperatura amostrada no exaustor. Fonte: Elaborado pelo Autor.}
  \label{fig:exaustor_temperatura}
\end{figure}


\begin{figure}[h!]
  \begin{center}
      \includegraphics[scale=0.55, page=4]{../resultados/img/resultados.pdf}
  \end{center}
  \caption{Linhas de base e sinais coletados em um dos 5 sensores na seleira universal. Fonte: Elaborado pelo Autor.}
  \label{fig:seleira_universal}
\end{figure}


\begin{figure}[h!]
  \begin{center}
      \includegraphics[scale=0.55, page=5]{../resultados/img/resultados.pdf}
  \end{center}
  \caption{Linhas de base e dados de velocidade [\SI{}{\milli\metre\per\second} no momento da instalação
  (A) x próximo da falha (B). Fonte: Elaborado pelo Autor.}
  \label{fig:seleira_universal_antes_depois}
\end{figure}


\begin{figure}[h!]
  \begin{center}
      \includegraphics[scale=0.55, page=6]{../resultados/img/resultados.pdf}
  \end{center}
  \caption{Temperatura [\SI{}{\celsius}] amostrada na seleira universal. Fonte: Elaborado pelo Autor.}
  \label{fig:seleira_universal_temperatura}
\end{figure}
% \begin{multline} 
% \int_0^{r_2}  F (r, \varphi ) dr d\varphi =  [\sigma r_2 / (2 \mu_0 )] \\
% \times   \int_0^{\infty} \exp(-\lambda |z_j - z_i |) \lambda^{-1} J_1 (\lambda  r_2 ) J_0 (\lambda r_i ) d\lambda 
% \label{eq:sample2}
% \end{multline}


\section{Conclusão}


\section*{Agradecimentos}


\bibliography{ifacconf}             % bib file to produce the bibliography                             

\end{document}
