%===============================================================================
% $Id: ifacconf.tex 19 2011-10-27 09:32:13Z jpuente $  
% Template for IFAC meeting papers
% Copyright (c) 2007-2008 International Federation of Automatic Control
%===============================================================================
\documentclass[a4paper]{ifacconf}
\usepackage{graphicx,amsmath,url}      % include this line if your document contains figures
\usepackage[round]{natbib}             % required for bibliography
\usepackage{siunitx}
\sisetup{locale = FR}
\usepackage{float}
\usepackage[table]{xcolor}
\usepackage{multirow}
\usepackage{makecell}
%===============================================================================


% ===============================================================
% Choose the language of the manuscript.
% If in English, choose 
% \def\portugues{0} 
%
% If in Portuguese or Spanish, choose
% \def\portugues{1} 
%
% Note that, if you are writing in Spanish, you need additional 
% adjusts in some parts of the text, which have been put in Portuguese only.
\def\portugues{1} 
% ===============================================================

% If the above line is commented, it is assumed manuscript in English:
\ifx\portugues\undefined
\def\portugues{0}
\fi


\if\portugues0
   \usepackage[english]{babel}
  \else
   \usepackage[spanish,brazil,english]{babel}
\fi

  

\usepackage[T1]{fontenc}
%\usepackage{inputenc}

\usepackage[utf8]{inputenc}

\usepackage{ae}


\if\portugues1
% =====================================================================
% =====================================================================
% If the manuscript is in Spanish, please change the texts adequatelly.
% You may also add other definitions in this part.
 \newtheorem{teorema}[thm]{{\em Teorema}}{ }
 \newtheorem{lema}[thm]{{\em Lema}}{ }
 \newtheorem{corolario}[thm]{{\em Corolário}}{ }
 \newenvironment{prova}{{\bf Prova.}}{ }
% ===============================================================
\fi

\begin{document}
	
	
\if\portugues1

% =====================================================================
% =====================================================================
% USE THIS PART IF THE TEXT IS IN PORTUGUES OR SPANISH
% =====================================================================
% If the manuscript is in Spanish, please change the texts adequately.
% =====================================================================
% 
\selectlanguage{brazil}
	
\begin{frontmatter}

\title{Desenvolvimento, Implementação e Estudo de Caso de um Software para Detecção Automática de Falhas em Motores Elétricos de Indução} 
% Title, preferably not more than 10 words.

\thanks[footnoteinfo]{Projeto realizado através de parceria entre a Universidade do Vale do Rio Dos Sinos - 
UNISINOS a empresa Floki Desenvolvimento de Sistemas Industriais, RS.}

\author[First]{Piaia, G. A.} 
\author[First]{Hoffmann, G.} 
\author[First]{Weber, L. O.}
\author[Second]{Figueiredo, R. M.}


\address[First]{Floki Desenvolvimento de Sistemas Industriais, RS (e-mails: piaia@flokisys.com; giu@flokisys.com; weber@flokisys.com).}
\address[Second]{Programa de Pós-Graduação em Engenharia Elétrica, Universidade do Vale do Rio Dos Sinos - 
UNISINOS, RS (e-mail: marquesf@unisinos.br).}


\selectlanguage{english}
\renewcommand{\abstractname}{{\bf Abstract:~}}
\begin{abstract}                % Abstract of not more than 250 words.
                        This work presents a solution that integrates software and hardware to solve the aforementioned problem,
                        including online and that learns the behaviour equipment, suggesting regions of alert and
                        danger, serving as a tool for taking decisions. For the development, the physical quantities of
                        vibration were used, together with signal processing techniques and data analysis. 
                        The solution was able to detected in advance in one case, and not the other, demonstrate that the equipment was in
                        severe condition.

\vskip 1mm% não altere esse espaçamento
\selectlanguage{brazil}
{\noindent \bf Resumo}: O presente trabalho propõe uma solução que integra \textit{software} e \textit{hardware} para
                        resolver o problema de se detectar falhas em motores elétricos de indução, inclusive em tempo real e que analisa o 
                        comportamento do equipamento, sugerindo regiões de alerta e perigo, servindo de ferramenta para tomada de
                        decisões. Para o desenvolvimento, as grandezas físicas de vibração foram modeladas
                        com técnicas de processamento de sinais e análise de dados. A solução 
                        foi empregada um estudo de caso em empresas de diferentes ramos da indústria, onde
                        o sistema detectou com antecedência em um dos casos, e no outro, demonstrou que o equipamento se
                        encontrava em estado severo.
\end{abstract}

\selectlanguage{english}


\begin{keyword}
  Vibration; Fault Analysis; Vibration and Temperature Analysis; Fault Detection; Online Monitoring; Induction Motor; Data Analysis.

\vskip 1mm% não altere esse espaçamento
\selectlanguage{brazil}
{\noindent\it Palavras-chaves:} Análise de Vibração e temperatura; Detecção de Falhas; Monitoramento em tempo real; Motores elétricos de indução; 
Análise de Dados.
\end{keyword}


\selectlanguage{brazil}


\end{frontmatter}
\else
% ===============================================================
% ===============================================================
% USE THIS PART IF THE TEXT IS IN ENGLISH
% ===============================================================
% ===============================================================
% 

\begin{frontmatter}

\title{Style for SBA Conferences \& Symposia: Use Title Case for
  Paper Title\thanksref{footnoteinfo}} 
% Title, preferably not more than 10 words.

\thanks[footnoteinfo]{Sponsor and financial support acknowledgment
goes here. Paper titles should be written in uppercase and lowercase
letters, not all uppercase.}

\author[First]{First A. Author} 
\author[Second]{Second B. Author, Jr.} 
\author[Third]{Third C. Author}


\address[First]{Faculdade de Engenharia Elétrica, Universidade do Triângulo, MG, (e-mail: autor1@faceg@univt.br).}
\address[Second]{Faculdade de Engenharia de Controle \& Automação, Universidade do Futuro, RJ (e-mail: autor2@feca.unifutu.rj)}
\address[Third]{Electrical Engineering Department, 
   Seoul National University, Seoul, Korea, (e-mail: author3@snu.ac.kr)}
   
\renewcommand{\abstractname}{{\bf Abstract:~}}   
   
\begin{abstract}                % Abstract of not more than 250 words.
These instructions give you guidelines for preparing papers for IFAC
technical meetings. Please use this document as a template to prepare
your manuscript. For submission guidelines, follow instructions on
paper submission system as well as the event website.
\end{abstract}

\begin{keyword}
Five to ten keywords, preferably chosen from the IFAC keyword list.
\end{keyword}

\end{frontmatter}
\fi

%===============================================================================
%===============================================================================
%===============================================================================


\section{Introdução}

Os motores elétricos de indução são amplamente empregados na indústria, pela sua confiabilidade e construção, o que torna uma opção 
barata e confiável \citep{Umans2003}. Mas com o tempo, estes equipamentos apresentam falhas decorrentes dos regimes de trabalho, esforços não 
projetados, ou somente pelo desgaste natural dos componentes. Nesses casos, a aplicação de uma estratégia de manutenção apropriada, que
monitore em tempo real e indique a manutenção em um tempo ótimo, pode diminuir em até 65\% os custos de manutenção \citep{Wu2013}. 

Uma técnica preditiva que possibilita essa intervenção ótima, pode ser desenvolvida com diversas ferramentas, proporcionando desde o monitoramento
de limites de vibração até o diagnóstico preciso da falha, indicando qual componente se encontra em falha. Indicar que a falha é
iminente e que está em desenvolvimento, é imprescindível para um processo ser mais produtivo, pois a manutenção pode ser programada
para um horário em que a máquina já estaria parada para outros fins, maximizando a produção. Além disso, o processo de manutenção é mais 
rápido, já que os elementos não terão se danificado por completo.

\begin{figure}[h!]
  \begin{center}
      \includegraphics[scale=.35]{../referencial/img/misadraw_analog_p2.png}
  \end{center}
  \caption{Ilustração com os tipos de desalinhamento. Fonte: \cite{Sopcik2019}.}
  \label{fig:misadraw_analog_p2}
\end{figure}

A vibração presente em motores elétricos de indução possui uma característica específica, depende do tipo de falha, cria uma assinatura
ao se analisar a vibração, sendo possível diagnosticar se o motor e o sistema em que ele está inserido está com boa saúde \citep{Wu2013}.
O desalinhamento ocorre quando o motor e o sistema não estão perfeitamente alinhados, podendo ser de dois tipos: paralelo e angular, os quais 
podem ser vistos, respectivamente na parte (a) e (b) da Figura \ref{fig:misadraw_analog_p2}.

Quando uma falha de desalinhamento está presente, há um aumento de até duas vezes nas harmônicas de alta 
frequência, como pode ser visto na Figura \ref{fig:misa_analog_p2} \cite{Sopcik2019}.

\begin{figure}[H]
  \begin{center}
      \includegraphics[scale=0.7, page=6]{../referencial/img/imagens_referencial.pdf}
  \end{center}
  \caption{Indicação espectral de desalinhamento na velocidade. Fonte: Adaptado de \cite{Sopcik2019}.}
  \label{fig:misa_analog_p2}
\end{figure}

Uma das primeiras técnicas para classificar o estado de saúde de uma máquina, é a norma ISO 10816-1, que recomenda níveis de vibração 
em valor eficaz de acordo com o porte da máquina, que pode ser visto na Tabela \ref{tab:iso10816-1_randall_p146}.

\begin{table}[H]
    \caption{Tabela de valores eficaz máximos de velocidade para cada porte de máquina indicados pela norma ISO 10816-1. Fonte: Adaptado de
      \cite{Wu2013}.}
    \label{tab:iso10816-1_randall_p146}
    \centering%
    \begin{minipage}{0.4833\textwidth}
        \begin{tabular*}{\textwidth}{|c|c|c|c|c|}
            \hline
            \multicolumn{1}{|c|}{\makecell{Velocidade \\ RMS [\SI{}{\milli\metre\per\second}]} }& Classe I                                                         & Classe II                                                        & Classe III                                                       & Classe IV                                                        \\ \hline
            \multicolumn{1}{|c|}{0.25}            & \multicolumn{1}{c|}{\cellcolor[HTML]{00FF02}}                    & \multicolumn{1}{c|}{\cellcolor[HTML]{00FF02}}                    & \multicolumn{1}{c|}{\cellcolor[HTML]{00FF02}}                    & \multicolumn{1}{c|}{\cellcolor[HTML]{00FF02}}                    \\ \cline{1-1}
            \multicolumn{1}{|c|}{0.45}            & \multicolumn{1}{c|}{\cellcolor[HTML]{00FF02}}                    & \multicolumn{1}{c|}{\cellcolor[HTML]{00FF02}}                    & \multicolumn{1}{c|}{\cellcolor[HTML]{00FF02}}                    & \multicolumn{1}{c|}{\cellcolor[HTML]{00FF02}}                    \\ \cline{1-1}
            \multicolumn{1}{|c|}{0.71}            & \multicolumn{1}{c|}{\multirow{-3}{*}{\cellcolor[HTML]{00FF02}A}} & \multicolumn{1}{c|}{\cellcolor[HTML]{00FF02}}                    & \multicolumn{1}{c|}{\cellcolor[HTML]{00FF02}}                    & \multicolumn{1}{c|}{\cellcolor[HTML]{00FF02}}                    \\ \cline{1-2}
            \multicolumn{1}{|c|}{1.12}            & \multicolumn{1}{c|}{\cellcolor[HTML]{00D2CB}}                    & \multicolumn{1}{c|}{\multirow{-4}{*}{\cellcolor[HTML]{00FF02}A}} & \multicolumn{1}{c|}{\cellcolor[HTML]{00FF02}}                    & \multicolumn{1}{c|}{\cellcolor[HTML]{00FF02}}                    \\ \cline{1-1} \cline{3-3}
            \multicolumn{1}{|c|}{1.8}             & \multicolumn{1}{c|}{\multirow{-2}{*}{\cellcolor[HTML]{00D2CB}B}} & \multicolumn{1}{c|}{\cellcolor[HTML]{00D2CB}}                    & \multicolumn{1}{c|}{\multirow{-5}{*}{\cellcolor[HTML]{00FF02}A}} & \multicolumn{1}{c|}{\cellcolor[HTML]{00FF02}}                    \\ \cline{1-2} \cline{4-4}
            \multicolumn{1}{|c|}{2.8}             & \multicolumn{1}{c|}{\cellcolor[HTML]{F8FF00}}                    & \multicolumn{1}{c|}{\multirow{-2}{*}{\cellcolor[HTML]{00D2CB}B}} & \multicolumn{1}{c|}{\cellcolor[HTML]{00D2CB}}                    & \multicolumn{1}{c|}{\multirow{-6}{*}{\cellcolor[HTML]{00FF02}A}} \\ \cline{1-1} \cline{3-3} \cline{5-5} 
            \multicolumn{1}{|c|}{4.5}             & \multicolumn{1}{c|}{\multirow{-2}{*}{\cellcolor[HTML]{F8FF00}C}} & \multicolumn{1}{c|}{\cellcolor[HTML]{F8FF00}}                    & \multicolumn{1}{c|}{\multirow{-2}{*}{\cellcolor[HTML]{00D2CB}B}} & \multicolumn{1}{c|}{\cellcolor[HTML]{00D2CB}}                    \\ \cline{1-2} \cline{4-4}
            \multicolumn{1}{|c|}{7.1}             & \multicolumn{1}{c|}{\cellcolor[HTML]{FE0000}}                    & \multicolumn{1}{c|}{\multirow{-2}{*}{\cellcolor[HTML]{F8FF00}C}} & \multicolumn{1}{c|}{\cellcolor[HTML]{F8FF00}}                    & \multicolumn{1}{c|}{\multirow{-2}{*}{\cellcolor[HTML]{00D2CB}B}} \\ \cline{1-1} \cline{3-3} \cline{5-5} 
            \multicolumn{1}{|c|}{11.2}            & \multicolumn{1}{c|}{\cellcolor[HTML]{FE0000}}                    & \multicolumn{1}{c|}{\cellcolor[HTML]{FE0000}}                    & \multicolumn{1}{c|}{\multirow{-2}{*}{\cellcolor[HTML]{F8FF00}C}} & \multicolumn{1}{c|}{\cellcolor[HTML]{F8FF00}}                    \\ \cline{1-1} \cline{4-4}
            \multicolumn{1}{|c|}{18}              & \multicolumn{1}{c|}{\cellcolor[HTML]{FE0000}}                    & \multicolumn{1}{c|}{\cellcolor[HTML]{FE0000}}                    & \multicolumn{1}{c|}{\cellcolor[HTML]{FE0000}}                    & \multicolumn{1}{c|}{\multirow{-2}{*}{\cellcolor[HTML]{F8FF00}C}} \\ \cline{1-1} \cline{5-5} 
            \multicolumn{1}{|c|}{28}              & \multicolumn{1}{c|}{\cellcolor[HTML]{FE0000}}                    & \multicolumn{1}{c|}{\cellcolor[HTML]{FE0000}}                    & \multicolumn{1}{c|}{\cellcolor[HTML]{FE0000}}                    & \multicolumn{1}{c|}{\cellcolor[HTML]{FE0000}}                    \\ \cline{1-1} \cline{5-5} 
            \multicolumn{1}{|c|}{45}              & \multicolumn{1}{c|}{\multirow{-5}{*}{\cellcolor[HTML]{FE0000}D}} & \multicolumn{1}{c|}{\multirow{-4}{*}{\cellcolor[HTML]{FE0000}D}} & \multicolumn{1}{c|}{\multirow{-3}{*}{\cellcolor[HTML]{FE0000}D}} & \multicolumn{1}{c|}{\multirow{-2}{*}{\cellcolor[HTML]{FE0000}D}} \\ \cline{1-1} \cline{5-5}
        \end{tabular*}
    \end{minipage}
  \end{table}

Onde

\begin{itemize}
    \item A = bom estado
    \item B = aceitável
    \item C = apenas tolerável
    \item D = não permitido
    \item Classe I = pequenas máquinas (potência menor que $\SI{15}{\kilo\watt}$)
    \item Classe II = máquinas médias sem uma fundação especial (potência entre $\SI{15}{\kilo\watt}$ e $\SI{75}{\kilo\watt}$)
    \item Classe III = máquinas grandes sobre uma fundação rígida e pesada
    \item Classe IV = máquinas grandes sobre uma fundação flexível (turbomáquinas) 
\end{itemize}

O trabalho de \cite*{Caesarendra2017}, é o qual realizou uma pesquisa sistemática com as principais técnicas de detecção de 
falhas que são estatísticas. Neste trabalho foram analisadas ferramentas como Kurtosis, Variância e valor RMS. As técnicas foram aplicadas em 
dados gerados em laboratório, e resultados foram elencados. Levanta a questão que o valor RMS é relevante para detectar iminentes falhas, mas
limitado para se diagnosticar falhas. Outro trabalho importante \cite*{Hemmati2016a} trata deste assunto também. No artigo, características de um sinal de vibração são levantadas e comparadas,
onde uma delas é o estudo do valor RMS. Uma das principais conclusões do trabalho, é que o 
valor RMS é o melhor parâmetro estatístico para detectar uma falha incipiente. Como podemos ver na literatura e estado da arte, é possível 
relacionar o Valor RMS com a saúde de um equipamento. O valor RMS, que é a representação mais comum de se descrever a energia que um sinal  
$x(t)$ carrega em qualquer período \citep{Cryer2010}, que pode ser representado da seguinte forma:

\begin{equation}\label{eq:rms_int}
    X_{RMS} = \sqrt{\frac{1}{T}\int_{0}^{T}{x(t)^2dt}}
\end{equation}

A ideia de se caracterizar a energia de um sinal através do Valor RMS \citep{Cryer2010}, vem da seguinte expressão:

\begin{equation}\label{eq:energia}
    E = \int_{0}^{T}{[x(t)]^2dt}
\end{equation}

Onde $E$ é a energia carregada pelo sinal x(t) \citep{Oppenheim2016}. Expandindo, se calcularmos a potência dissipada em um resistor de
$\SI{1}{\ohm}$ num intervalo de tempo T, temos a seguinte expressão:

\begin{equation}\label{eq:power}
    P = \frac{1}{T}\int_{0}^{T}{[x(t)]^2dt}
\end{equation}

Se compararmos as equações \ref{eq:rms_int} e \ref{eq:power}, nota-se que o valor RMS está relacionado com a raiz quadrada da potência do 
sinal. Ao se utilizar sinais discretos, o valor RMS pode ser representado da seguinte forma:

\begin{equation}\label{eq:rms_disc}
    X_{RMS} = \sqrt{\frac{1}{N}\sum_{i=1}^{N}{x^2(i)}}
\end{equation}

Teoria da probabilidade é uma ferramenta matemática sólida
para quantificar e manipular incertezas, formando a fundamentação para o reconhecimento de padrões. O estudo da 
distribuição de probabilidades em variáveis contínuas é essencial para entender a dinâmica destes dados. A distribuição de probabilidades mais
conhecida é a Gaussiana, que também é chamada de distribuição normal \citep{Andersen1986}. Três dos principais parâmetros para se analisar
em uma distribuição normal são: média, variância e desvio padrão, onde a média de um sinal $x(t)$ num período T, pode ser calculado da seguinte
forma \citep{Dinardo}:

\begin{equation}\label{eq:X_lim}
    \bar{x} = \lim_{T\rightarrow\infty}{\frac{1}{T^*}} \int_{0}^{T}{x(t)dt}
\end{equation}

Já a variância:

\begin{equation}\label{eq:variancia}
    \sigma^2 = \lim_{T^{*}\rightarrow\infty}{\frac{1}{T^*}} \int_{0}^{T}{[x(t)-\bar{x}]^2dt}
\end{equation}

Onde $\sigma$ é o desvio padrão, que representa a dispersão, em outras palavras, o quão distante os dados amostrados estão da média. Um dado
amostrado que se encontra disperso, ou seja, com elevado desvio padrão, pode significar ganho ou perda de energia, e se esse dado
for um valor RMS de vibração, pode caracterizar uma anomalia. Assim que os conceitos básicos foram apresentados, podemos prosseguir para a proposta do trabalho.


%===============================================================================

\section{Solução Proposta}

Este trabalho tem como principal objetivo a implementação de uma técnica de manutenção preditiva, onde possibilite a detecção
de falhas via análise da vibração em motores elétricos de indução e sistemas mancalizados. A solução une técnicas de processamento de 
sinais e aprendizado de máquinas, com o objetivo de identificar limites de vibração para detecção de falhas de forma menos invasiva e mais
econômica. 

A proposta do presente trabalho começa com a utilização de um simulador de falhas em máquinas, no qual foi possível modelar a 
proposta de solução de \textit{software}. Passa pelo desenvolvimento do \textit{software}, que emprega ferramentas como Python, InfluxDB e 
Node-Red para a prototipação. 
Na integração entre \textit{software} e \textit{hardware}, é unida a solução de software ao \textit{hardware} industrial, através de um 
protocolo industrial. Ao final da execução, temos o estudo de caso, o qual aplica a solução desenvolvida em duas empresas de diferentes setores,
com o objetivo de prova de conceito e validação do produto, gerando possível aquisição do sistema. 

O diagrama da Figura \ref{fig:fluxo_integracao} representa o fluxo básico dos sinais, indo do sensor até o sistema embarcado, o qual processa, armazena e 
disponibiliza a visualização em tempo real dos dados.

\begin{figure}[H]
  \begin{center}
      \includegraphics[scale=0.53, page=1]{../metodologia/img/fluxo_layout.pdf}
  \end{center}
  \caption{Diagrama de integração  entre sensores e \textit{software}. Fonte: Adaptado pelo autor.}
  \label{fig:fluxo_integracao}
\end{figure}

O sistema embarcado, o qual é a unidade de processamento da aplicação, é composto por uma Raspberry Pi 3 Modelo B+, sendo um dispositivo muito 
conhecido e confiável, executando uma distribuição Linux baseada em Debian.

Para uma interface entre homem e máquina, utilizamos a ferramenta 
Node-Red para prototipar a presente aplicação.

\begin{figure}[H]
  \begin{center}
      \includegraphics[scale=0.53, page=2]{../metodologia/img/fluxo_layout.pdf}
  \end{center}
  \caption{Fluxo de acesso do \textit{software}.} 
  \label{fig:fluxo_software}
\end{figure}

A Figura \ref{fig:fluxo_software} mostra a estrutura básica das telas do software, que é uma ferramenta 
baseada em web que pode ser acessada por qualquer dispositivo com navegador que esteja na rede.

%===============================================================================
\section{Metodologia}

O desenvolvimento do projeto, começa pela configuração do simulador de falhas MFS\textsuperscript\textregistered,
da empresa SpectraQuest\textsuperscript\textregistered, até a implementação em campo. A metodologia do desenvolvimento se encontra na
Figura \ref{fig:metodologia}, onde as etapas estão descritas em forma de fluxo de trabalho.

\begin{figure}[H]
  \begin{center}
      \includegraphics[scale=1, page=3]{../metodologia/img/software.pdf}
      \caption{Metodologia do Projeto.}
  \end{center} 
  \label{fig:metodologia}
\end{figure}

Como dito anteriormente no fluxo de desenvolvimento, fez se o uso de simulações antes da implementação do sistema.
Para isso, foi utilizado um simulador de falhas em motores elétricos de indução e sistemas mancalizados, denominado
MFS \textsuperscript \textregistered (\textit{Machinery Fault Simulator} - simulador de falhas em máquinas), que pode ser visto na Figura 
\ref{fig:real_plant}.

\begin{figure}[H]
    \begin{center}
        \includegraphics[scale=.25]{../metodologia/img/real_plant.jpeg}
    \end{center}
    \caption{Estrutura do simulador MFS\textsuperscript\textregistered. Fonte: \cite{SpectraQuest2011}.}
    \label{fig:real_plant}
\end{figure}

O simulador é composto por um motor elétrico de indução trifásico de 1 HP, o qual está conectado com um eixo via um 
acoplamento. O eixo possui dois discos dourados e furados, onde é possível se colocar cargas para criar um desbalanceamento no sistema.

Progredindo no desenvolvimento, para a aquisição das grandezas físicas, as quais alimentam a aplicação de software, foi selecionado o sensor 
de vibração e temperatura QM30VT2\textsuperscript\textregistered, da Banner Engineering\textsuperscript\textregistered, que possui as seguintes características:

\begin{table}[H]
  \caption{Especificações do Sensor QM30VT2\textsuperscript\textregistered.}
  \label{tab:espec_sensor}
  \centering
  \begin{minipage}{.33\textwidth}
    \begin{tabular*}{\textwidth}{c|c}
      \hline
      Especificação            & Valores                                     \\ \hline
      \hline
      Intervalo de medição     &  0 a $\SI{46}{\milli\metre\per\second}$     \\
      Largura de banda         &  10 a $\SI{4}{\kilo\hertz}$                 \\ 
      Acurácia                 &  $\mp 10 \%$ a $\SI{25}{\celsius}$          \\
      Amostragem               &  $\SI{20}{\kilo\hertz}$                     \\
      Tempo de Amostragem      &  $\SI{0.4}{\second}$                        \\
      Grau de Certificação     &  IP67                                       \\ \hline
    \end{tabular*}
  \end{minipage}
\end{table}

De acordo com a tabela \ref{tab:espec_sensor}, ele possui uma largura de banda de até $\SI{4}{\kilo\hertz}$, adequado até para detecção de falhas em rolamentos, como 
descrito no referencial teórico. Os dados são adquiridos via Modbus RTU (\textit{Remote Terminal Unit} - Unidade de terminal remoto) diretamente
do sensor, ou ainda, via Modbus TCP (\textit{Transmission Control Protocol} - Protocolo de Controle de Transmissão) a partir de um gateway
\textit{wireless} e um nó, facilitando a implantação por não utilizar cabos entre os sensores e a unidade de processamento. 

Na tarefa de aquisição de dados, a qual foi desenvolvida utilizando a linguagem de programação Python, o protocolo utilizado foi o Modbus, 
muito comum na indústria e de fácil utilização, principalmente porque a leitura é feita de forma periódica e com um intervalo de até 
\SI{5}{\minute} entre cada uma, que é um dos conceitos básicos do protocolo. O banco de dados utilizado é o InfluxDB, um banco de dados não 
relacional para armazenamento de séries temporais, que também é uma característica dos dados amostrados.
Já na tarefa de geração automáticas dos limites de vibração, sendo a principal contribuição do presente trabalho, ela manipula banco de dados e 
variáveis de ambiente da aplicação para encontrar as linhas de base e os limiares de alerta e perigo, basicamente encontrando as seguintes 
relações:

\begin{equation}\label{eq:ml}
    \bar{X}_{n+1} = \bar{X}_{n} + \frac{x + \bar{X}_n}{n}
\end{equation}

Onde $\bar{X}_{n+1}$ é a média aritmética para os $n+1$ elementos, $ \bar{X}_{n}$ é a média aritmética anterior,
$x$ é o novo dado adquirido. Já a relação da variância, pode ser calculada da seguinte forma:


\begin{equation}\label{eq:ml2}
    n \cdot var_{n+1} = var_{n} + (x - \bar{X}_{n+1})(x-\bar{X}_n)
\end{equation}

E por fim, o desvio padrão $\sigma_{n+1}$ para um novo elemento:
\begin{equation}\label{eq:ml3}
    \sigma_{n+1} = \sqrt{\frac{var}{n}}
\end{equation}

Assumimos que a região de alerta ($A$), a qual significa uma moderada tendência à falha, pode ser calculada da seguinte forma: 
\begin{equation}\label{eq:ml4}
    A = \bar{X}_{n+1} + \alpha \sigma_{n+1} 
\end{equation}

Onde $\alpha$ é um coeficiente ajustável. Segundo a mesma regra, a região onde há uma falha iminente ($Pe$), um comportamento ambiental ou regime de funcionamento atípico, 
pode ser calculado da seguinte forma:

\begin{equation}\label{eq:ml5}
    Pe = \bar{X}_{n+1} + \beta \sigma_{n+1} 
\end{equation}

Onde $\beta$ é outro coeficiente ajustável, que deve ser maior do que $\alpha$. Durante o processo de geração automática dos limites, o valor de
$n$ varia de 1 até 100 amostras, restando apenas uma reta após o processo de geração, sendo o regime de monitoramento. Após o desenvolvimento
da solução, temos a etapa de estudo de caso, que se encontra na sequência.


\subsection{Estudo de Caso}

Como descrito no fluxograma do projeto, a etapa final do desenvolvimento é a de testes em campo, em forma de estudo de caso, com o objetivo 
de analisar a robustez do produto, da comunicação entre o sensor e a unidade de processamento e da confiabilidade da aplicação de \textit{software}. 
Para aplicar a solução, foram escolhidos equipamentos reais e funcionais, que operam 24 horas, 7 dias por semana em empresas parceiras,
com o objetivo de prova de conceito e posterior aquisição pelas mesmas. Para isto, um exaustor industrial foi escolhido em uma delas, que é 
uma montadora da região metropolitana de Porto Alegre - RS. A Figura \ref{fig:exautor} representa um exaustor industrial, um dispositivo que com
o decorrer do uso, apresenta severas falhas com o acúmulo de partículas que estão suspensas no ar. 
Uma falha num dispositivo como este, acarreta numa parada de  produção, impactando diretamente no financeiro da empresa, justificando 
facilmente o investimento da tecnologia desenvolvida neste trabalho.

\begin{figure}[h!]
  \begin{center}
      \includegraphics[scale=1]{../metodologia/img/exaustor.png}
  \end{center}
  \caption{Fotografia externa de um exaustor industrial. Fonte: Fornecida pela empresa.}
  \label{fig:exautor}
\end{figure}

Em um destes dispositivos, foi instalado um sensor industrial de vibração e temperatura, com o objetivo de coletar, processar e armazenar os
dados, criando um \textit{dashboard} com o estado de saúde do motor elétrico de indução. A Figura \ref{fig:sensor_exaustor} apresenta o sensor instalado.
O motor já se encontrava em uso há algum tempo antes dos testes em campo, significando que já existe uma base de vibração maior do que se fosse
um motor há pouco recondicionado, exigindo um ajuste manual na ferramenta de geração automáticas dos limites de vibração, reduzindo os limites 
criados pelo \textit{software}, que consideram que o motor está em perfeito estado de saúde para começar a criação dos limites. 
Já estava previsto na etapa de análise de mercado, a possibilidade de se ajustar os resultados da geração automáticas dos limites de vibração, 
devido a grande frequência de instalações em motores que possuem médio ou elevado MTBF (\textit{Mean Time Between Failures} - tempo médio 
entre falhas), mas que ainda oferecem prejuízos se a falha ocorrer de forma imprevista.

\begin{figure}[h!]
  \begin{center}
      \includegraphics[scale=1]{../metodologia/img/sensor_exaustor.jpg}
  \end{center}
  \caption{Fotografia externa do sensor instalado em um exaustor industrial. Fonte: Fornecida pela empresa.}
  \label{fig:sensor_exaustor}
\end{figure}

Na Figura anterior, o sensor está fixo de forma magnética na carcaça do motor, facilitando a instalação do mesmo. Já as 
características dos testes podem ser vistas na abela \ref{tab:exautor}. 

\begin{table}[H]
    \caption{Características da prova de conceito no exaustor.}
    \label{tab:exautor}
    \centering%
    \begin{minipage}{.4\textwidth}
      \begin{tabular*}{\textwidth}{c|c}
        \hline
        Característica                          & Valores                                    \\ \hline
        \hline
        Tempo entre cada amostragem             &  $\SI{5}{\minute}$                         \\
        Número de sensores                      &  1                                         \\ 
        Potência do motor                       &  $\SI{22.065}{\kilo\watt}$                      \\
        Classe do motor segundo a  ISO 10816-1  &  II                                         \\
      \end{tabular*} 
    \end{minipage}
  \end{table}


Já na outra empresa, uma produtora de tabaco e seus derivados, também no estado do Rio Grande do Sul, o sistema foi instalado numa 
seleira universal, não especificamente em um motor elétrico de indução, mas em 5 pontos estratégicos, distribuindo os 5 sensores nos principais pontos 
de falhas: acoplamentos, moto-redutores e nas proximidades de polias. A seleira universal tem um papel muito importante na produção, sendo um dos gargalos, pois qualquer falha imprevista, também impacta na produção diária, prejudicando toda a produção. A Tabela \ref{tab:seleira_universal} apresenta 
as características da prova de conceito na seleira universal.

\begin{table}[H]
    \caption{Características da prova de conceito na seleira universal.}
    \label{tab:seleira_universal}
    \centering%
    \begin{minipage}{.4\textwidth}
      \begin{tabular*}{\textwidth}{c|c}
        \hline
        Característica                          & Valores                                    \\ \hline
        \hline
        Tempo entre cada amostragem             &  $\SI{30}{\second}$                        \\
        Número de sensores                      &  5                                         \\ 
        Potência do motor                       &  $\SI{1.1185}{\kilo\watt}$                 \\
        Classe do motor segundo a  ISO 10816-1  &  I                                         \\
      \end{tabular*}
    \end{minipage}
  \end{table}

  Conforme a tabela, este cenário possui um motor elétrico de potência substancialmente menor do que no primeiro, além de uma amostragem maior.
%===============================================================================

\section{Resultados}

Como se trata de um estudo de caso, e para as empresas, uma prova de conceito, os resultados também foram divididos por equipamento avaliado:
exaustor e seleira universal. Inicialmente são apresentados capturas de telas do \textit{software} na parte de análise dos sinais 
coletados, para fim de dar uma visão global do estado de saúde do equipamento no dado momento. Para a apresentação dos resultados do exaustor,
foram feitas capturas de tela do \textit{software} do sistema que estava instalado no motor, de forma remota e supervisionada pelos 
responsáveis pela manutenção dos mesmos. A Figura \ref{fig:exaustor_1} é a captura de tela do sistema no dia em que um relatório sobre a saúde 
dos exaustores foi gerado, com os dados daquele momento, e com os limites gerados automaticamente no momento da instalação (06/12/2020).

\begin{figure*}[h!]
  \begin{center}
      \includegraphics[scale=0.75, page=1]{../resultados/img/resultados.pdf}
  \end{center}
  \caption{Captura do lado direito da tela do sistema instalado no exaustor.}
  \label{fig:exaustor_1}
\end{figure*}


A captura de tela está com os dados de velocidade e aceleração do eixo Z. Fica claro que as grandezas físicas
ultrapassam os limites estabelecidos pelo \textit{software} somente no estado de alerta, mas isso não significa que o motor esteja em bom estado.
Se consultarmos a Tabela \ref{tab:iso10816-1_randall_p146}, veremos que os valores para um motor de Classe II, segundo norma  ISO 10816-1, está 
entre apenas tolerável e não permitido, dependendo do regime de funcionamento. 
Se olharmos o eixo z, que se encontra na Figura \ref{fig:exaustor_xz}, os valores se encontram em uma faixa não permitida, deixando claro o risco 
de falha iminente do motor.

\begin{figure}[H]
  \begin{center}
      \includegraphics[scale=0.55, page=2]{../resultados/img/resultados.pdf}
  \end{center}
  \caption{Velocidade RMS [\SI{}{\milli\metre\per\second}] coletados do eixo x (A) e z (B).}
  \label{fig:exaustor_xz}
\end{figure}

O eixo z se encontra em estado severo de vibração, não entrando em alarme devido à geração dos alertas tardio executado
pelo o operador, que agora se torna uma tarefa de gerenciamento do sistema, de se executar da maneira mais adequada, que é logo após o recondicionamento 
ou ajustar manualmente os limites de acordo com a norma ISO 10816-1. Outra característica importante é a temperatura, que pode ser vista na 
Figura \ref{fig:exaustor_temperatura}, que teve uma dinâmica com característica de um sistema linear de primeira ordem.


\begin{figure}[H]
  \begin{center}
      \includegraphics[scale=0.55, page=3]{../resultados/img/resultados.pdf}
  \end{center}
  \caption{Temperatura amostrada no exaustor.}
  \label{fig:exaustor_temperatura}
\end{figure}

Fica nítido também o processo de aquecimento do dispositivo, seguindo os ciclos de funcionamento, sendo regulares. O estudo de caso do 
exaustor se mostrou um ótimo cenário para se analisar o comportamento do sistema num dispositivo que já estava em funcionamento por um bom tempo 
sem um recondicionamento, onde a técnica de geração de alertas, como esperado, não contribuiu com o objetivo de se detectar falhas, pois as
condições iniciais não eram as ideais e os limites ficaram além dos permitidos. Mas o funcionamento geral da ferramenta foi excelente, sem nenhum
problema grave reportado, funcionando nas especificações e dentro do esperado pela empresa parceira. Um relatório foi gerado e entregue
para a empresa, contendo todos os dados e explicações, o qual teve aceite e aquisição da solução.

A segunda prova de conceito foi realizada em uma seleira universal, como descrito anteriormente, onde 5 sensores foram distribuídos pela máquina, com o 
objetivo de monitorar os pontos críticos. Como no caso do exaustor, foram realizadas capturas de telas do sistema. Esta máquina não apresenta
um processo contínuo como o exaustor, exigindo uma amostragem maior para se captar todas as características do funcionamento. 
Os sensores foram instalados no dia 13/11/2020, mas os dados começaram a ser coletados apenas no dia 16/11/2020, quando a 
ferramenta de geração automática de limites foi usada e os alertas foram criados. 
A ferramenta foi aplicada em todos os 5 sensores, que foram instalados em acoplamentos, próximo de polias e moto-redutores.
Cada um monitorou um comportamento diferente, mas o sensor mais importante se mostrou ser o 5, onde os resultados foram mais expressivos. 
Dentre os outros sensores, pelo menos mais 3 apresentaram comportamento anômalos, mas nenhum como este. A Figura \ref{fig:seleira_universal_antes_depois}
apresenta uma comparação dos dados de velocidade logo após a instalação, com os dados na iminência de uma falha.


\begin{figure}[h!]
  \begin{center}
      \includegraphics[scale=0.55, page=5]{../resultados/img/resultados.pdf}
  \end{center}
  \caption{Linhas de base e dados de velocidade [\SI{}{\milli\metre\per\second} no momento da instalação
  (A) x próximo da falha (B).}
  \label{fig:seleira_universal_antes_depois}
\end{figure}


Os dados coletados na iminência da falha ultrapassam os limites de alerta e perigo, mesmo que não continuamente, mas
como a regra estabelecida na metodologia, da existência de 5 picos ou média acima dos limites, caracterizando um alerta. Como são muitos dados,
um a cada $\SI{30}{\second}$, a visualização das retas de limites ficam comprometidas, e só é possível ver pela parte da tela onde se edita os
valores, ou pelos pontos azuis-escuros, amarelos e vermelhos no lado direito do gráfico. Além disso, é possível notar que no início do processo,
o sistema já se encontrava em uma situação pouco tolerável, segundo a norma ISO 10816-1, isso porquê estão instalados em partes que 
normalmente vibram mais, justamente por possuírem muitos elementos mecânicos e partes móveis. Por fim, a seleira universal apresentou o rompimento de uma
correia alguns dias após a coleta de dados, e visualmente alertar uma condição severa de falha, consolidando o uso da ferramenta em
mais um equipamento, independente de ser um motor ou outro sistema mecânico. A aplicação funcionou perfeitamente, similar ao estudo de caso do
exaustor, apresentando apenas uma falha de salvar repetidas vezes os limites, que foi resolvida remotamente e incorporada ao \textit{software} do projeto.
Uma peculiaridade de um sistema mancalizado, que difere de uma instalação em um motor elétrico de indução, é o comportamento da temperatura,
que pode ser visto na Figura \ref{fig:seleira_universal_temperatura}. 

\begin{figure}[h!]
  \begin{center}
      \includegraphics[scale=0.55, page=6]{../resultados/img/resultados.pdf}
  \end{center}
  \caption{Temperatura [\SI{}{\celsius}] amostrada na seleira universal.}
  \label{fig:seleira_universal_temperatura}
\end{figure}

Ela tem um perfil mais discreto, e não seguindo um sistema de primeira ordem, por não estar próximo de uma significativa fonte de calor, por isso
os limites gerados para algo com pouca variância, são limites muito próximos dos dados amostrados, ocorrendo falsos positivos apenas com a
variação da temperatura ambiente. A solução para este problema, já prevista na metodologia do projeto, é a possibilidade do operador alterar os limites, de 
acordo com a dinâmica que o sistema está inserido, não cabendo utilizar a ferramenta automática de geração de limites onde há pouca variância 
ou o equipamento já se encontra em severo desgaste.

\section{Conclusão e Trabalhos Futuros}

A aplicação do protótipo em campo no estudo de caso, mostrou como é diversificada a indústria em que estamos inseridos. 
A necessidade de diferentes amostragens, configuração manual dos limites de alerta e perigo, e diferentes regimes de funcionamento dos
equipamentos, também evidencia quão flexível a ferramenta desenvolvida necessita ser para atender estes cenários. O emprego de ferramentas de 
rápida prototipação, como o uso de Node-Red, Raspberry Pi e um sistema de coleta de dados robusto e consolidado no mercado, possibilitou essa 
flexibilidade ao presente projeto, que saiu de uma ideia, até um produto que já está sendo comercializado.

Na análise dos resultados ficou clara a aplicabilidade via a análise dos estudos de casos. O sistema foi
capaz de monitorar, armazenar e de alarmar um estado crítico, na iminência de uma falha em um dos estudos. Já no outro, que se trata do exaustor,
o sistema foi instalado tardiamente, cabendo apenas avaliar segundo a norma ISO 10816-1. Outra característica importante, é a utilização da
ferramenta de geração automática de limites de vibração, que se mostrou muito eficaz em criar linhas de alarme e perigo, quando o sistema é instalado em um dispositivo
que há pouco passou por manutenção, e no uso para grandezas com variância considerável. Porém, demonstrou menor eficácia nos seguintes cenários:
máquinas que já estão há tempo funcionando sem manutenção e grandezas com pouca variância, ficando a cargo do usuário avaliar se o uso da 
geração automática se aplica ou não. O presente trabalho também gerou um registro de software, que no momento da publicação deste  documento,
se encontra em andamento.

Trabalhos futuros a partir desse devem ser desenvolvidos, principalmente na adição de parâmetros de entrada, que utilizem a norma ISO 10816-1
por padrão. Também é necessário tornar o \textit{software} escalável e executável de forma remota, para garantir que os dados não serão comprometidos. Analisar estudos de 
casos de mais máquinas e motores, melhorando o modelo da aplicação. Por fim, trabalhar no desenvolvimento de técnicas de diagnóstico de falhas,
agregando valor ao sistema.

\bibliography{ifacconf}             % bib file to produce the bibliography                             

\end{document}
